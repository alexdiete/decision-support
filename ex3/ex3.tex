\documentclass[11pt]{article}
\usepackage{amsmath}
\usepackage{caption}

\usepackage{algorithmicx}
\usepackage{verbatim}
\usepackage{algpseudocode}
\usepackage{algorithm}

\usepackage{graphicx}
\usepackage{listings}

\title{Exercisesheet No.3}
\author{Alexander Diete \and Magnus M\"uller \and Martin Pfannem\"uller}

\begin{document}
\maketitle

\section*{Ex.1}

a)
In the card scenario an atomic event is the drawing of one specific card. As the deck consists of 52 cards we have 52 atomic events. \\
\\
b) The probability of drawing one specific card is 1 divided by the amount of cards, thus: 
$$\frac{1}{52}$$ \\
\\
c)
As there are as many black cards as there are red cards, 26 out of the total of 52 cards are black. Hence the probability is:
$$\frac{26}{52} = \frac{1}{2}$$ \\

\section*{Ex.2}

a) $\binom{52}{5} = 2598960$ \\
\\
b) $\frac{1}{\binom{52}{5}} = \frac{1}{2598960}$ \\
\\
c) \\
\\
i) There are four Royal Straigth Flushes (Heart, Spades, Clubs, Diamonds). Thus, the answer is: $4 * \frac{1}{2598960}$ \\
\\
ii) There are 13 possibilities of a Four of a kind. Since one card does not matter, every remaining card is fine. $\frac{13*(52-4)}{2598960}$

\section*{Ex.3}
a)\\
i) $P(\neg a) = P(\Omega \setminus a) = P(\Omega) - P(a) = 1 - P(a)$ \\
ii) TODO:  $P(a \wedge \neg b) = P(a \cap (\Omega \setminus b))$ \\
\\
b) If a and b are disjoint then: $P(a \cup b) = P(a) + P(b) = \frac{1}{3} + \frac{5}{6} > 1$ This cannot be the case since $P(\Omega) = 1$. Thus, a and be are not disjoint. \\

\section*{Ex.4}

a)\\
i) TODO: $P(a) = \frac{P(a \wedge \neg b)}{P(\neg b)} $\\
ii) TODO\\
\\
b)\\
\\
i) At first Alice has the probability of $\frac{1}{2}$ to win. When Alice does not win, it is Rob's turn. When he loses (also with P = $\frac{1}{2}$), Alice has again the possiblity to win with $\frac{1}{2}$ resulting in $\frac{1}{2}+\frac{1}{2}*\frac{1}{2}*\frac{1}{2}$. The third step is that, Alice does not win in the first round, Rob does not win in his first try, Alice does not win on her second try, Rob does not win on his second try and Alice wins on her third try. This results in $\frac{1}{2}+\frac{1}{2}*\frac{1}{2}*\frac{1}{2}+\frac{1}{2}*\frac{1}{2}*\frac{1}{2}*\frac{1}{2}*\frac{1}{2}$. This goes on forever. This can be described with the following sum: $ \sum\nolimits_{i=0}^\infty(\frac{1}{2^{2i-1}})$. This sum converges to the probability P(AliceWins) = $\frac{2}{3}$ TODO: Wie ausrechen (gezogen von Wolfram Alpha).\\
\\
ii) P(head) =: p; P(AliceWins) = $p + (1-p) *(1-p) * p + (1-p) * (1-p) * (1-p) * (1-p) *p + \dots$ \\
\\
iii) On every flip the second person's probability to win is decreased by half compared to firsts. E.g. The first persons first flip has the probability of $\frac{1}{2}$ to win, the second person has the probability of $\frac{1}{2}*\frac{1}{2}$ to win. The same applies for the other flips. Thus, we would flip first.\\
\\
\section*{Ex.5}
a) P(toothache)=$0.108+0.012+0.016+0.064=\frac{1}{5}$\\
b) P(catch)=$0.108+0.016+0.072+0.144=0.34$\\
c) P(cavity$|$catch)=$0.108+0.072=0.18$\\
d) P(toothache $\vee$ catch) = $0.108+0.016+0.012+0.064+0.072+0.144=0.416\\
$P(cavity$|$toothache $\vee$ catch)=$(0.108+0.012+0.072)/0.416=0.0.4615$\\

\section*{Ex.6}

Given:\\
p: (virus) present\\
r: recoqnized\\
A: $P_A(r|p)=0.95; P_A(r|\neg p)=0.1$\\
B: $P_B(r|p)=0.9; P_B(r|\neg p)=0.05$\\
$P(p)=0.01$\\
\\
Alex warum falsch (siehe Lösungsbuch S. 117):\\
$P_A(p|r)=\frac{P_A(r|p)*P_A(p)}{P_A(r)}=\frac{P_A(r|p)*P_A(p)}{P_A(r|\neg p)+P_A(r|p)}=\frac{0.95*0.01}{0.95+0.1}=0.0090$\\
$P_B(p|r)=\frac{P_B(r|p)*P_B(p)}{P_B(r)}=\frac{P_B(r|p)*P_B(p)}{P_B(r|\neg p)+P_B(r|p)}=\frac{0.9*0.01}{0.9+0.05}=0.00947$\\

\section*{Ex.7}

First we setup the basic formula:
$$P(B|j,m) = \alpha P(B) \cdot \sum_{e}P(e) \cdot \sum_{a}P(a|B,e) \cdot P(j|a) \cdot P(m|a)$$
Afterwards the probabilities are replaces with factors:
$$P(B|j,m) = \alpha f1(B) \times \sum_{e}f2(E) \times \sum_{a}f3(A,B,E) \times f4(A) \times f5(A)$$
We can sum out f3, f4 and f5 as they all share A. First we combine f4 and f5 (and call it f6):
$$ f4(A) = 
\begin{pmatrix}
  P(j|a) \\
  P(j|\neg a)
\end{pmatrix}
 = 
\begin{pmatrix}
  0.9 \\
  0.05
\end{pmatrix}
$$ 
 
$$ f5(A) = 
\begin{pmatrix}
  P(m|a) \\
  P(m|\neg a)
\end{pmatrix}
 = 
\begin{pmatrix}
  0.7 \\
  0.01
\end{pmatrix}
$$ 

$$ f6(A) = 
\begin{pmatrix}
  P(j|a)*P(m|a) \\
  P(j|\neg a)*P(m|\neg a)
\end{pmatrix}
 = 
\begin{pmatrix}
  0.9 \cdot 0.7 \\
  0.05 \cdot 0.01
\end{pmatrix}
 = 
\begin{pmatrix}
  0.63 \\
  0.0005
\end{pmatrix}
$$ 
Then we can sum out A in f3 and combine it with f6. We call this factor f7:
$$
  f7(B,E) = f3(a,B,E) * 0.63 + f3(\neg a,B,E) * 0.0005
$$

$$
  f3(a,B,E) = 
\begin{pmatrix}
  P(a|b,e) & P(a|b,\neg e) \\
  P(a|\neg b,e) & P(a|\neg b,\neg e)
\end{pmatrix} \cdot 0.63
=
\begin{pmatrix}
  0.95 & 0.94 \\
  0.29 & 0.001
\end{pmatrix} \cdot 0.63
= 
\begin{pmatrix}
  0.60 & 0.59 \\
  0.17 & 0.00063
\end{pmatrix} 
$$
$$
  f3(\neg a,B,E) = 
\begin{pmatrix}
  P(\neg a|b,e) & P(\neg a|b,\neg e) \\
  P(\neg a|\neg b,e) & P(\neg a|\neg b,\neg e)
\end{pmatrix} \cdot 0.0005
=
\begin{pmatrix}
  0.05 & 0.06 \\
  0.71 & 0.999
\end{pmatrix} \cdot 0.0005
$$
\end{document}